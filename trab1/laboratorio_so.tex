\section{Laboratório de Sistemas Operacionais}

\subsection{Aplicações de Pesquisa}

\subsubsection{Contribuições Open\--Source}

Laboratórios de Universidades comumente contribuem com projetos
Open\--Source.

Nesse sentido, o laboratório de sistemas operacionais pode desenvolver novas funcionalidades, corrigir \textit{bugs} ou prover melhorias de desempenho para sistemas operacionais como Linux, FreeBSD ou MINIX.

Além disso, o laboratório pode desenvolver algum \textit{port} para um
hardware específico ou ainda não suportado.

O laboratório pode também estar desenvolvendo o driver para
algum hardware específico, como uma placa de vídeo, por exemplo.

\subsubsection{Protótipos de Sistemas Operacionais}

Um dos livros mais utilizados em aulas de Sistemas Operacionais, que,
na verdade, é nosso livro\--texto atualmente, e, é o livro\--texto
da disciplina há anos em diversas faculdades do mundo, é o livro
``Sistemas Operacionais Modernos'' do Tanembaum.

Para a primeira edição desse livro, o autor, na falta de uma implementação
open\--source de um sistema operacional, desenvolveu o MINIX, em cima
do UNIX, como um sistema operacional educacional e de pesquisa.

E, em cima desse protótipo, foi que Linus Torvalds desenvolveu o Linux.

Assim, uma das aplicações do laboratório será desenvolver novos protótipos
de sistemas operacionais, talvez seguindo as tendências de microkernel
ou de novas melhorias em termos de segurança e confiabilidade.

\subsubsection{Pesquisa em Sistemas Operacionais \-- Novas aplicações e hardware}

Essas tendências citadas acima, como aplicações de microkernel (que, na
verdade, é a pesquisa do Tanembaum), assim como a ideia do MINIX 3
de um sistema regenerativo poderiam ser exemplos de novas aplicações
desenvolvidas no laboratório.

Além disso, o laboratório pode desenvolver novas técnicas e algoritmos,
sejam eles de escalonamento, gerenciamento de memória, sistemas de
arquivos, drivers de dispositivos, etc.

\subsubsection{Análise de Desempenho}

Dadas as outras aplicações do labatório, é natural que sejam
desenvolvidas ferramentas de suporte.

Dentre as ferramentes, podem ser citadas ferramentas de medição
de parâmetros avaliativos, como parâmetros de desempenho.

Assim, a eficiência de aplicações desenvolvidas pode ser mensurada,
além de ser possível otimizar a pesquisa a partir dessas informações
e dados obtidos.

\subsubsection{Virtualização e Containerização}

Mais um paralelo com os aprendizados em Sistemas Operacionais,
vimos que, para que seja suportada virtualização, uma série de requisitos
deve ser obedecida pelo sistema operacional, e, consequentemente, também
deve ser suportada pelo hardware.

Além disso, com a popularização de tecnologias de containerização,
e o destaque que tem sido dado para tecnologias de virtualização,
é esperado que o laboratório estude melhorias nessas áreas.

Isso pode acontecer seja desenvolvendo monitores de máquina virtual,
supervisores, plataformas de containerização, além de, é claro,
suporte em algum modelo de sistema operacional já existente.

\subsubsection{Segurança e Computação Confiável}

Outro tópico que pode ser desenvolvido no laboratório, envolve segurança,
como, também, a ideia do MINIX 3 de fazer apenas um microkernel, com
partes do sistema operacional como drivers, rodando em user level.

Além disso, o laboratório pode desenvolver técnicas de detecção e
mitigação de vulnerabilidades, como, por exemplo, técnicas de boot
seguro, controle de acesso, detecção de intrusão e protocolos de
comunicação segura.

Outro exemplo poderia ser implementar ideias de compilação reprodutível,
como o Nix, ou, o sistema operacional que envolve esse gerenciador
de software: NixOS.

\subsubsection{Sistemas Embarcados}

Muitos sistemas operacionais são desenvolvidos para uso em sistemas
embarcados, como dispositivos IoT, eletrodomésticos inteligentes e
sistemas automotivos.

Laboratórios de universidades podem desenvolver sistemas operacionais
em tempo real e leves, otimizados para esses ambientes com recursos
limitados.

Também vimos que isso estará relacionado, por exemplo, aos algoritmos
de escalonamento, que podem envolver um método cooperativo (não-preemptivo).

\subsubsection{Educação e Treinamento}

O laboratório também pode desempenhar um papel crucial na educação
dos alunos sobre sistemas operacionais, novamente relacionando ao caso
do Tanembaum.

O laboratório poderá fornecer experiência prática, realizar workshops
e oferecer cursos para ajudar os alunos a entender os sistemas operacionais
e promover a inovação nessa área.

Além disso, pode atrair novos alunos para a área de pesquisa, seja
como IC's no Laboratório, ou como mestrandos ou doutorados.

\newpage
\subsection{Qualidade de Serviço}

Este laboratório, diferentemente de Laboratórios como o de Computação Gráfica,
também descrito neste artigo, possui uma dependência significativamente
maior de uma infraestrutura de rede.

Para aplicações como educação, segurança ou virtualização, é mais
visível a necessidade de se garantir a qualidade de serviço.

Portanto, mapearemos os seguintes parâmetros de QoS às aplicações do laboratório:

\begin{enumerate}
    \item Latência (Delay)
    \item Variação de Latência (Jitter)
    \item Vazão (Throughput)
    \item Perda de Pacotes (Packet Loss)
\end{enumerate}

\subsubsection{Contribuições Open\--Source}

\begin{itemize}
    \item Vazão: dependendo do tamanho do projeto a que se realiza
        as contribuições, vazão passa a ser um parâmetro cuja
        garantia é necessária.
\end{itemize}

\subsubsection{Protótipos de Sistemas Operacionais}

Como prototipar SO's envolve testagem e validação de \textit{features}
como protocolos de rede, \textit{drivers}, virtualização, etc., é
necessário suporte de QoS específico.

\begin{itemize}
    \item Latência: suporte a medição de latência.
    \item Variação de Latência: suporte a medição de variação de latência.
    \item Vazão: suporte a medição de vazão.
    \item Perda de Pacotes: suporte a medição de perda de pacotes.
\end{itemize}

\subsubsection{Pesquisa em Sistemas Operacionais \-- Novas aplicações e hardware}

Assim como em prototipagem, aqui pode haver a necessidade de se criar
\textit{payloads} de teste para validar novas \textit{features} de
sistemas operacionais.

Nesse sentido, é necessário suporte de QoS para garantir que os
testes sejam válidos.

\begin{itemize}
    \item Latência: suporte a medição de latência.
    \item Variação de Latência: suporte a medição de variação de latência.
    \item Vazão: suporte a medição de vazão.
    \item Perda de Pacotes: suporte a medição de perda de pacotes.
\end{itemize}

\subsubsection{Análise de Desempenho}

Aqui, além dos critérios mais obrigatórios, como suporte a medição,
outras garantias de QoS também são necessárias.

\begin{itemize}
    \item Latência: para que as medições sejam válidas e precisas,
        é necessário que o \textit{overhead} de medição seja mínimo.
        Assim, deve\--se garantir baixa latência.
    \item Vazão: para analisar aplicações de alto desempenho,
        precisa\--se garantir suporte a altas vazões.
    \item Perda de Pacotes: novamente, para garantir precisão
        de ferramentas, é necessário, no mínimo, garantir
        uma perda constante de pacotes.
\end{itemize}

\subsubsection{Virtualização e Containerização}

\begin{itemize}
    \item Latência: para que a virtualização não seja perceptível
        ao usuário, é necessário garantir baixa latência.
    \item Variação de Latência: para melhor experiência do usuário nas máquinas virtuais em rede
        é necessário garantir baixo \textit{jitter}.
\end{itemize}

\subsubsection{Segurança e Computação Confiável}

Essa é outra atividade que é mais independente de rede.

No entanto, considerando casos de uso de rede, como, por exemplo,
detecções de intrusão em uma máquina em rede, ou, então, comunicação
segura entre máquinas virtuais via rede, podemos propor medidas de QoS.

Outro caso que envolve rede indiretamente é o caso da compilação reprodutível,
em que, pode\--se demandar maior vazão por ter de se compilar localmente
algum pacote, ao invés de realizar \textit{download} de um pacote binário.

\begin{itemize}
    \item Latência: para que seja possível realizar a detecção
        de intrusão em tempo hábil, é necessário garantir baixa latência.
    \item Vazão: para garantir que e possível realizar a compilação
        de um pacote em tempo hábil, é necessário garantir suporte
        a altas vazões.
\end{itemize}

\subsubsection{Sistemas Embarcados}

Aqui temos outro caso atípico, dado que, em sistemas embarcados, quando é necessário
algum tipo de conectividade, costuma\--se ser realizado por Lora\--Wan ou algum
tipo de solução adaptada para o caso de uso.

Mas, pensando no desenvolvimento de pesquisa em sistemas embarcados, podemos
adotar um ponto de vista que atenderá a maior parte dos casos de uso, quando
trata\--se de Qos.

\begin{itemize}
    \item Latência: dado que sistemas embarcados costumam ser utilizados em
        aplicações de monitoramento de alguma variável, é necessário garantir
        baixa latência.
    \item Perda de Pacotes: dado que nestes sistemas é menos comum implementar\--se
        algum tipo de controle de fluxo, é necessário garantir baixa perda de pacotes.
\end{itemize}

\subsubsection{Educação e Treinamento}

\begin{itemize}
    \item Latência: dado que muitas das acoes educacionais do laboratorio
        podem envolver videoconferências, é necessário garantir baixa latência.
    \item Vazão: dada a qualidade de vídeo e áudio que se deseja
        para as videoconferências, é necessário garantir suporte
        a altas vazões.
\end{itemize}

\newpage
\subsection{Estimativas de QoS}

Valores estimados a partir das tabelas fornecidas no enunciado da ITU.
No entanto, valores adaptados para o contexto do laboratório e para
um contexto mais recente, onde a banda requisitada é maior.

\begin{table}[H]
    \centering
    \begin{tabular}{ccccc}
    \toprule
    \textbf{Aplicação} & \textbf{Latência} & \textbf{Variação de Latência} & \textbf{Vazão} & \textbf{Perda de Pacotes}\\
    & \footnotesize{ms} & \footnotesize{ms} & \footnotesize{Mbps} & \footnotesize{\%}\\
    \midrule
    Open Source & 10 & 5 & 15 & 0.1\\
    Prototipagem & 50 & 10 & 10 & 0.5\\
    Pesquisa & 20 & 5 & 20 & 0.2\\
    Benchmarking & 15 & 3 & 50 & 0.1\\
    Virtualização & 30 & 20 & 200 & 0.3\\
    Segurança & 20 & 10 & 10 & 0.5\\
    Embarcados & 10 & 5 & 5 & 0.1\\
    Educação & 30 & 5 & 200 & 0.4\\
    \bottomrule
    \end{tabular}
    \caption{Análise quantitativa de QoS para o Laboratório de Sistemas Operacionais}
    \label{tab:qos-so}
\end{table}
